
% Default to the notebook output style

    


% Inherit from the specified cell style.




    
\documentclass[11pt]{article}

    
    
    \usepackage[T1]{fontenc}
    % Nicer default font (+ math font) than Computer Modern for most use cases
    \usepackage{mathpazo}

    % Basic figure setup, for now with no caption control since it's done
    % automatically by Pandoc (which extracts ![](path) syntax from Markdown).
    \usepackage{graphicx}
    % We will generate all images so they have a width \maxwidth. This means
    % that they will get their normal width if they fit onto the page, but
    % are scaled down if they would overflow the margins.
    \makeatletter
    \def\maxwidth{\ifdim\Gin@nat@width>\linewidth\linewidth
    \else\Gin@nat@width\fi}
    \makeatother
    \let\Oldincludegraphics\includegraphics
    % Set max figure width to be 80% of text width, for now hardcoded.
    \renewcommand{\includegraphics}[1]{\Oldincludegraphics[width=.8\maxwidth]{#1}}
    % Ensure that by default, figures have no caption (until we provide a
    % proper Figure object with a Caption API and a way to capture that
    % in the conversion process - todo).
    \usepackage{caption}
    \DeclareCaptionLabelFormat{nolabel}{}
    \captionsetup{labelformat=nolabel}

    \usepackage{adjustbox} % Used to constrain images to a maximum size 
    \usepackage{xcolor} % Allow colors to be defined
    \usepackage{enumerate} % Needed for markdown enumerations to work
    \usepackage{geometry} % Used to adjust the document margins
    \usepackage{amsmath} % Equations
    \usepackage{amssymb} % Equations
    \usepackage{textcomp} % defines textquotesingle
    % Hack from http://tex.stackexchange.com/a/47451/13684:
    \AtBeginDocument{%
        \def\PYZsq{\textquotesingle}% Upright quotes in Pygmentized code
    }
    \usepackage{upquote} % Upright quotes for verbatim code
    \usepackage{eurosym} % defines \euro
    \usepackage[mathletters]{ucs} % Extended unicode (utf-8) support
    \usepackage[utf8x]{inputenc} % Allow utf-8 characters in the tex document
    \usepackage{fancyvrb} % verbatim replacement that allows latex
    \usepackage{grffile} % extends the file name processing of package graphics 
                         % to support a larger range 
    % The hyperref package gives us a pdf with properly built
    % internal navigation ('pdf bookmarks' for the table of contents,
    % internal cross-reference links, web links for URLs, etc.)
    \usepackage{hyperref}
    \usepackage{longtable} % longtable support required by pandoc >1.10
    \usepackage{booktabs}  % table support for pandoc > 1.12.2
    \usepackage[inline]{enumitem} % IRkernel/repr support (it uses the enumerate* environment)
    \usepackage[normalem]{ulem} % ulem is needed to support strikethroughs (\sout)
                                % normalem makes italics be italics, not underlines
    

    
    
    % Colors for the hyperref package
    \definecolor{urlcolor}{rgb}{0,.145,.698}
    \definecolor{linkcolor}{rgb}{.71,0.21,0.01}
    \definecolor{citecolor}{rgb}{.12,.54,.11}

    % ANSI colors
    \definecolor{ansi-black}{HTML}{3E424D}
    \definecolor{ansi-black-intense}{HTML}{282C36}
    \definecolor{ansi-red}{HTML}{E75C58}
    \definecolor{ansi-red-intense}{HTML}{B22B31}
    \definecolor{ansi-green}{HTML}{00A250}
    \definecolor{ansi-green-intense}{HTML}{007427}
    \definecolor{ansi-yellow}{HTML}{DDB62B}
    \definecolor{ansi-yellow-intense}{HTML}{B27D12}
    \definecolor{ansi-blue}{HTML}{208FFB}
    \definecolor{ansi-blue-intense}{HTML}{0065CA}
    \definecolor{ansi-magenta}{HTML}{D160C4}
    \definecolor{ansi-magenta-intense}{HTML}{A03196}
    \definecolor{ansi-cyan}{HTML}{60C6C8}
    \definecolor{ansi-cyan-intense}{HTML}{258F8F}
    \definecolor{ansi-white}{HTML}{C5C1B4}
    \definecolor{ansi-white-intense}{HTML}{A1A6B2}

    % commands and environments needed by pandoc snippets
    % extracted from the output of `pandoc -s`
    \providecommand{\tightlist}{%
      \setlength{\itemsep}{0pt}\setlength{\parskip}{0pt}}
    \DefineVerbatimEnvironment{Highlighting}{Verbatim}{commandchars=\\\{\}}
    % Add ',fontsize=\small' for more characters per line
    \newenvironment{Shaded}{}{}
    \newcommand{\KeywordTok}[1]{\textcolor[rgb]{0.00,0.44,0.13}{\textbf{{#1}}}}
    \newcommand{\DataTypeTok}[1]{\textcolor[rgb]{0.56,0.13,0.00}{{#1}}}
    \newcommand{\DecValTok}[1]{\textcolor[rgb]{0.25,0.63,0.44}{{#1}}}
    \newcommand{\BaseNTok}[1]{\textcolor[rgb]{0.25,0.63,0.44}{{#1}}}
    \newcommand{\FloatTok}[1]{\textcolor[rgb]{0.25,0.63,0.44}{{#1}}}
    \newcommand{\CharTok}[1]{\textcolor[rgb]{0.25,0.44,0.63}{{#1}}}
    \newcommand{\StringTok}[1]{\textcolor[rgb]{0.25,0.44,0.63}{{#1}}}
    \newcommand{\CommentTok}[1]{\textcolor[rgb]{0.38,0.63,0.69}{\textit{{#1}}}}
    \newcommand{\OtherTok}[1]{\textcolor[rgb]{0.00,0.44,0.13}{{#1}}}
    \newcommand{\AlertTok}[1]{\textcolor[rgb]{1.00,0.00,0.00}{\textbf{{#1}}}}
    \newcommand{\FunctionTok}[1]{\textcolor[rgb]{0.02,0.16,0.49}{{#1}}}
    \newcommand{\RegionMarkerTok}[1]{{#1}}
    \newcommand{\ErrorTok}[1]{\textcolor[rgb]{1.00,0.00,0.00}{\textbf{{#1}}}}
    \newcommand{\NormalTok}[1]{{#1}}
    
    % Additional commands for more recent versions of Pandoc
    \newcommand{\ConstantTok}[1]{\textcolor[rgb]{0.53,0.00,0.00}{{#1}}}
    \newcommand{\SpecialCharTok}[1]{\textcolor[rgb]{0.25,0.44,0.63}{{#1}}}
    \newcommand{\VerbatimStringTok}[1]{\textcolor[rgb]{0.25,0.44,0.63}{{#1}}}
    \newcommand{\SpecialStringTok}[1]{\textcolor[rgb]{0.73,0.40,0.53}{{#1}}}
    \newcommand{\ImportTok}[1]{{#1}}
    \newcommand{\DocumentationTok}[1]{\textcolor[rgb]{0.73,0.13,0.13}{\textit{{#1}}}}
    \newcommand{\AnnotationTok}[1]{\textcolor[rgb]{0.38,0.63,0.69}{\textbf{\textit{{#1}}}}}
    \newcommand{\CommentVarTok}[1]{\textcolor[rgb]{0.38,0.63,0.69}{\textbf{\textit{{#1}}}}}
    \newcommand{\VariableTok}[1]{\textcolor[rgb]{0.10,0.09,0.49}{{#1}}}
    \newcommand{\ControlFlowTok}[1]{\textcolor[rgb]{0.00,0.44,0.13}{\textbf{{#1}}}}
    \newcommand{\OperatorTok}[1]{\textcolor[rgb]{0.40,0.40,0.40}{{#1}}}
    \newcommand{\BuiltInTok}[1]{{#1}}
    \newcommand{\ExtensionTok}[1]{{#1}}
    \newcommand{\PreprocessorTok}[1]{\textcolor[rgb]{0.74,0.48,0.00}{{#1}}}
    \newcommand{\AttributeTok}[1]{\textcolor[rgb]{0.49,0.56,0.16}{{#1}}}
    \newcommand{\InformationTok}[1]{\textcolor[rgb]{0.38,0.63,0.69}{\textbf{\textit{{#1}}}}}
    \newcommand{\WarningTok}[1]{\textcolor[rgb]{0.38,0.63,0.69}{\textbf{\textit{{#1}}}}}
    
    
    % Define a nice break command that doesn't care if a line doesn't already
    % exist.
    \def\br{\hspace*{\fill} \\* }
    % Math Jax compatability definitions
    \def\gt{>}
    \def\lt{<}
    % Document parameters
    \title{example\_of\_parameter\_sweep\_concept}
    
    
    

    % Pygments definitions
    
\makeatletter
\def\PY@reset{\let\PY@it=\relax \let\PY@bf=\relax%
    \let\PY@ul=\relax \let\PY@tc=\relax%
    \let\PY@bc=\relax \let\PY@ff=\relax}
\def\PY@tok#1{\csname PY@tok@#1\endcsname}
\def\PY@toks#1+{\ifx\relax#1\empty\else%
    \PY@tok{#1}\expandafter\PY@toks\fi}
\def\PY@do#1{\PY@bc{\PY@tc{\PY@ul{%
    \PY@it{\PY@bf{\PY@ff{#1}}}}}}}
\def\PY#1#2{\PY@reset\PY@toks#1+\relax+\PY@do{#2}}

\expandafter\def\csname PY@tok@w\endcsname{\def\PY@tc##1{\textcolor[rgb]{0.73,0.73,0.73}{##1}}}
\expandafter\def\csname PY@tok@c\endcsname{\let\PY@it=\textit\def\PY@tc##1{\textcolor[rgb]{0.25,0.50,0.50}{##1}}}
\expandafter\def\csname PY@tok@cp\endcsname{\def\PY@tc##1{\textcolor[rgb]{0.74,0.48,0.00}{##1}}}
\expandafter\def\csname PY@tok@k\endcsname{\let\PY@bf=\textbf\def\PY@tc##1{\textcolor[rgb]{0.00,0.50,0.00}{##1}}}
\expandafter\def\csname PY@tok@kp\endcsname{\def\PY@tc##1{\textcolor[rgb]{0.00,0.50,0.00}{##1}}}
\expandafter\def\csname PY@tok@kt\endcsname{\def\PY@tc##1{\textcolor[rgb]{0.69,0.00,0.25}{##1}}}
\expandafter\def\csname PY@tok@o\endcsname{\def\PY@tc##1{\textcolor[rgb]{0.40,0.40,0.40}{##1}}}
\expandafter\def\csname PY@tok@ow\endcsname{\let\PY@bf=\textbf\def\PY@tc##1{\textcolor[rgb]{0.67,0.13,1.00}{##1}}}
\expandafter\def\csname PY@tok@nb\endcsname{\def\PY@tc##1{\textcolor[rgb]{0.00,0.50,0.00}{##1}}}
\expandafter\def\csname PY@tok@nf\endcsname{\def\PY@tc##1{\textcolor[rgb]{0.00,0.00,1.00}{##1}}}
\expandafter\def\csname PY@tok@nc\endcsname{\let\PY@bf=\textbf\def\PY@tc##1{\textcolor[rgb]{0.00,0.00,1.00}{##1}}}
\expandafter\def\csname PY@tok@nn\endcsname{\let\PY@bf=\textbf\def\PY@tc##1{\textcolor[rgb]{0.00,0.00,1.00}{##1}}}
\expandafter\def\csname PY@tok@ne\endcsname{\let\PY@bf=\textbf\def\PY@tc##1{\textcolor[rgb]{0.82,0.25,0.23}{##1}}}
\expandafter\def\csname PY@tok@nv\endcsname{\def\PY@tc##1{\textcolor[rgb]{0.10,0.09,0.49}{##1}}}
\expandafter\def\csname PY@tok@no\endcsname{\def\PY@tc##1{\textcolor[rgb]{0.53,0.00,0.00}{##1}}}
\expandafter\def\csname PY@tok@nl\endcsname{\def\PY@tc##1{\textcolor[rgb]{0.63,0.63,0.00}{##1}}}
\expandafter\def\csname PY@tok@ni\endcsname{\let\PY@bf=\textbf\def\PY@tc##1{\textcolor[rgb]{0.60,0.60,0.60}{##1}}}
\expandafter\def\csname PY@tok@na\endcsname{\def\PY@tc##1{\textcolor[rgb]{0.49,0.56,0.16}{##1}}}
\expandafter\def\csname PY@tok@nt\endcsname{\let\PY@bf=\textbf\def\PY@tc##1{\textcolor[rgb]{0.00,0.50,0.00}{##1}}}
\expandafter\def\csname PY@tok@nd\endcsname{\def\PY@tc##1{\textcolor[rgb]{0.67,0.13,1.00}{##1}}}
\expandafter\def\csname PY@tok@s\endcsname{\def\PY@tc##1{\textcolor[rgb]{0.73,0.13,0.13}{##1}}}
\expandafter\def\csname PY@tok@sd\endcsname{\let\PY@it=\textit\def\PY@tc##1{\textcolor[rgb]{0.73,0.13,0.13}{##1}}}
\expandafter\def\csname PY@tok@si\endcsname{\let\PY@bf=\textbf\def\PY@tc##1{\textcolor[rgb]{0.73,0.40,0.53}{##1}}}
\expandafter\def\csname PY@tok@se\endcsname{\let\PY@bf=\textbf\def\PY@tc##1{\textcolor[rgb]{0.73,0.40,0.13}{##1}}}
\expandafter\def\csname PY@tok@sr\endcsname{\def\PY@tc##1{\textcolor[rgb]{0.73,0.40,0.53}{##1}}}
\expandafter\def\csname PY@tok@ss\endcsname{\def\PY@tc##1{\textcolor[rgb]{0.10,0.09,0.49}{##1}}}
\expandafter\def\csname PY@tok@sx\endcsname{\def\PY@tc##1{\textcolor[rgb]{0.00,0.50,0.00}{##1}}}
\expandafter\def\csname PY@tok@m\endcsname{\def\PY@tc##1{\textcolor[rgb]{0.40,0.40,0.40}{##1}}}
\expandafter\def\csname PY@tok@gh\endcsname{\let\PY@bf=\textbf\def\PY@tc##1{\textcolor[rgb]{0.00,0.00,0.50}{##1}}}
\expandafter\def\csname PY@tok@gu\endcsname{\let\PY@bf=\textbf\def\PY@tc##1{\textcolor[rgb]{0.50,0.00,0.50}{##1}}}
\expandafter\def\csname PY@tok@gd\endcsname{\def\PY@tc##1{\textcolor[rgb]{0.63,0.00,0.00}{##1}}}
\expandafter\def\csname PY@tok@gi\endcsname{\def\PY@tc##1{\textcolor[rgb]{0.00,0.63,0.00}{##1}}}
\expandafter\def\csname PY@tok@gr\endcsname{\def\PY@tc##1{\textcolor[rgb]{1.00,0.00,0.00}{##1}}}
\expandafter\def\csname PY@tok@ge\endcsname{\let\PY@it=\textit}
\expandafter\def\csname PY@tok@gs\endcsname{\let\PY@bf=\textbf}
\expandafter\def\csname PY@tok@gp\endcsname{\let\PY@bf=\textbf\def\PY@tc##1{\textcolor[rgb]{0.00,0.00,0.50}{##1}}}
\expandafter\def\csname PY@tok@go\endcsname{\def\PY@tc##1{\textcolor[rgb]{0.53,0.53,0.53}{##1}}}
\expandafter\def\csname PY@tok@gt\endcsname{\def\PY@tc##1{\textcolor[rgb]{0.00,0.27,0.87}{##1}}}
\expandafter\def\csname PY@tok@err\endcsname{\def\PY@bc##1{\setlength{\fboxsep}{0pt}\fcolorbox[rgb]{1.00,0.00,0.00}{1,1,1}{\strut ##1}}}
\expandafter\def\csname PY@tok@kc\endcsname{\let\PY@bf=\textbf\def\PY@tc##1{\textcolor[rgb]{0.00,0.50,0.00}{##1}}}
\expandafter\def\csname PY@tok@kd\endcsname{\let\PY@bf=\textbf\def\PY@tc##1{\textcolor[rgb]{0.00,0.50,0.00}{##1}}}
\expandafter\def\csname PY@tok@kn\endcsname{\let\PY@bf=\textbf\def\PY@tc##1{\textcolor[rgb]{0.00,0.50,0.00}{##1}}}
\expandafter\def\csname PY@tok@kr\endcsname{\let\PY@bf=\textbf\def\PY@tc##1{\textcolor[rgb]{0.00,0.50,0.00}{##1}}}
\expandafter\def\csname PY@tok@bp\endcsname{\def\PY@tc##1{\textcolor[rgb]{0.00,0.50,0.00}{##1}}}
\expandafter\def\csname PY@tok@fm\endcsname{\def\PY@tc##1{\textcolor[rgb]{0.00,0.00,1.00}{##1}}}
\expandafter\def\csname PY@tok@vc\endcsname{\def\PY@tc##1{\textcolor[rgb]{0.10,0.09,0.49}{##1}}}
\expandafter\def\csname PY@tok@vg\endcsname{\def\PY@tc##1{\textcolor[rgb]{0.10,0.09,0.49}{##1}}}
\expandafter\def\csname PY@tok@vi\endcsname{\def\PY@tc##1{\textcolor[rgb]{0.10,0.09,0.49}{##1}}}
\expandafter\def\csname PY@tok@vm\endcsname{\def\PY@tc##1{\textcolor[rgb]{0.10,0.09,0.49}{##1}}}
\expandafter\def\csname PY@tok@sa\endcsname{\def\PY@tc##1{\textcolor[rgb]{0.73,0.13,0.13}{##1}}}
\expandafter\def\csname PY@tok@sb\endcsname{\def\PY@tc##1{\textcolor[rgb]{0.73,0.13,0.13}{##1}}}
\expandafter\def\csname PY@tok@sc\endcsname{\def\PY@tc##1{\textcolor[rgb]{0.73,0.13,0.13}{##1}}}
\expandafter\def\csname PY@tok@dl\endcsname{\def\PY@tc##1{\textcolor[rgb]{0.73,0.13,0.13}{##1}}}
\expandafter\def\csname PY@tok@s2\endcsname{\def\PY@tc##1{\textcolor[rgb]{0.73,0.13,0.13}{##1}}}
\expandafter\def\csname PY@tok@sh\endcsname{\def\PY@tc##1{\textcolor[rgb]{0.73,0.13,0.13}{##1}}}
\expandafter\def\csname PY@tok@s1\endcsname{\def\PY@tc##1{\textcolor[rgb]{0.73,0.13,0.13}{##1}}}
\expandafter\def\csname PY@tok@mb\endcsname{\def\PY@tc##1{\textcolor[rgb]{0.40,0.40,0.40}{##1}}}
\expandafter\def\csname PY@tok@mf\endcsname{\def\PY@tc##1{\textcolor[rgb]{0.40,0.40,0.40}{##1}}}
\expandafter\def\csname PY@tok@mh\endcsname{\def\PY@tc##1{\textcolor[rgb]{0.40,0.40,0.40}{##1}}}
\expandafter\def\csname PY@tok@mi\endcsname{\def\PY@tc##1{\textcolor[rgb]{0.40,0.40,0.40}{##1}}}
\expandafter\def\csname PY@tok@il\endcsname{\def\PY@tc##1{\textcolor[rgb]{0.40,0.40,0.40}{##1}}}
\expandafter\def\csname PY@tok@mo\endcsname{\def\PY@tc##1{\textcolor[rgb]{0.40,0.40,0.40}{##1}}}
\expandafter\def\csname PY@tok@ch\endcsname{\let\PY@it=\textit\def\PY@tc##1{\textcolor[rgb]{0.25,0.50,0.50}{##1}}}
\expandafter\def\csname PY@tok@cm\endcsname{\let\PY@it=\textit\def\PY@tc##1{\textcolor[rgb]{0.25,0.50,0.50}{##1}}}
\expandafter\def\csname PY@tok@cpf\endcsname{\let\PY@it=\textit\def\PY@tc##1{\textcolor[rgb]{0.25,0.50,0.50}{##1}}}
\expandafter\def\csname PY@tok@c1\endcsname{\let\PY@it=\textit\def\PY@tc##1{\textcolor[rgb]{0.25,0.50,0.50}{##1}}}
\expandafter\def\csname PY@tok@cs\endcsname{\let\PY@it=\textit\def\PY@tc##1{\textcolor[rgb]{0.25,0.50,0.50}{##1}}}

\def\PYZbs{\char`\\}
\def\PYZus{\char`\_}
\def\PYZob{\char`\{}
\def\PYZcb{\char`\}}
\def\PYZca{\char`\^}
\def\PYZam{\char`\&}
\def\PYZlt{\char`\<}
\def\PYZgt{\char`\>}
\def\PYZsh{\char`\#}
\def\PYZpc{\char`\%}
\def\PYZdl{\char`\$}
\def\PYZhy{\char`\-}
\def\PYZsq{\char`\'}
\def\PYZdq{\char`\"}
\def\PYZti{\char`\~}
% for compatibility with earlier versions
\def\PYZat{@}
\def\PYZlb{[}
\def\PYZrb{]}
\makeatother


    % Exact colors from NB
    \definecolor{incolor}{rgb}{0.0, 0.0, 0.5}
    \definecolor{outcolor}{rgb}{0.545, 0.0, 0.0}



    
    % Prevent overflowing lines due to hard-to-break entities
    \sloppy 
    % Setup hyperref package
    \hypersetup{
      breaklinks=true,  % so long urls are correctly broken across lines
      colorlinks=true,
      urlcolor=urlcolor,
      linkcolor=linkcolor,
      citecolor=citecolor,
      }
    % Slightly bigger margins than the latex defaults
    
    \geometry{verbose,tmargin=1in,bmargin=1in,lmargin=1in,rmargin=1in}
    
    

    \begin{document}
    
    
    \maketitle
    
    

    
    \hypertarget{demo-of-how-to-conduct-a-parameter-sweep}{%
\subsection{\#\#\# Demo of how to conduct a parameter
sweep}\label{demo-of-how-to-conduct-a-parameter-sweep}}

    \hypertarget{problem-such-and-such}{%
\section{Problem such and such}\label{problem-such-and-such}}

\hypertarget{problem-statement}{%
\subsection{Problem statement}\label{problem-statement}}

Here I state the nature of the question that this document is intending
to help to answer

\hypertarget{methodology}{%
\subsection{Methodology}\label{methodology}}

\hypertarget{parameter-swep-on-this-and-that}{%
\subsubsection{Parameter swep on this and
that}\label{parameter-swep-on-this-and-that}}

In here a parameter sweep on parameter this is conducted, while keeping
that constant.

A metric of A,B,and C is evaluated and plotted against D.

    \hypertarget{import-libraries}{%
\paragraph{import libraries}\label{import-libraries}}

    \begin{Verbatim}[commandchars=\\\{\}]
{\color{incolor}In [{\color{incolor} }]:} \PY{c+c1}{\PYZsh{}\PYZsh{} Imports}
        
        \PY{k+kn}{import} \PY{n+nn}{sys}
        \PY{k+kn}{import} \PY{n+nn}{numpy} \PY{k}{as} \PY{n+nn}{np}
        \PY{k+kn}{import} \PY{n+nn}{h5py} \PY{c+c1}{\PYZsh{} HDF data file format package}
        
        \PY{c+c1}{\PYZsh{} path to the public handybeam}
        \PY{n}{sys}\PY{o}{.}\PY{n}{path}\PY{o}{.}\PY{n}{append}\PY{p}{(}\PY{l+s+s2}{\PYZdq{}}\PY{l+s+s2}{../.}\PY{l+s+s2}{\PYZdq{}}\PY{p}{)}
        \PY{c+c1}{\PYZsh{} path to the secrets}
        \PY{n}{sys}\PY{o}{.}\PY{n}{path}\PY{o}{.}\PY{n}{append}\PY{p}{(}\PY{l+s+s2}{\PYZdq{}}\PY{l+s+s2}{../../secrets/.}\PY{l+s+s2}{\PYZdq{}}\PY{p}{)}
        
        \PY{k+kn}{import} \PY{n+nn}{handybeam}
        \PY{k+kn}{import} \PY{n+nn}{handybeam}\PY{n+nn}{.}\PY{n+nn}{world}
        \PY{k+kn}{import} \PY{n+nn}{handybeam}\PY{n+nn}{.}\PY{n+nn}{tx\PYZus{}array\PYZus{}library}
        \PY{k+kn}{import} \PY{n+nn}{handybeam}\PY{n+nn}{.}\PY{n+nn}{visualise}
        \PY{k+kn}{import} \PY{n+nn}{handybeam}\PY{n+nn}{.}\PY{n+nn}{samplers}\PY{n+nn}{.}\PY{n+nn}{rectilinear\PYZus{}sampler} \PY{k}{as} \PY{n+nn}{rect\PYZus{}sampler}
        \PY{k+kn}{import} \PY{n+nn}{handybeam}\PY{n+nn}{.}\PY{n+nn}{samplers}\PY{n+nn}{.}\PY{n+nn}{clist\PYZus{}sampler} \PY{k}{as} \PY{n+nn}{clist\PYZus{}sampler}
        \PY{k+kn}{from} \PY{n+nn}{handybeam}\PY{n+nn}{.}\PY{n+nn}{solver} \PY{k}{import} \PY{n}{Solver}
        \PY{k+kn}{from} \PY{n+nn}{handybeam}\PY{n+nn}{.}\PY{n+nn}{evaluator} \PY{k}{import} \PY{n}{Evaluator}
        
        \PY{k+kn}{import} \PY{n+nn}{secrets}\PY{n+nn}{.}\PY{n+nn}{secret\PYZus{}module1}
\end{Verbatim}


    \hypertarget{initialize-the-simulated-world}{%
\subsection{\#\#\# Initialize the simulated
world}\label{initialize-the-simulated-world}}

    \begin{Verbatim}[commandchars=\\\{\}]
{\color{incolor}In [{\color{incolor} }]:} \PY{c+c1}{\PYZsh{} Initialise the world }
        
        \PY{c+c1}{\PYZsh{} NEW: add a frequency parameter to the world \PYZhy{} so that the kernels are compiled with correct constants.}
        \PY{c+c1}{\PYZsh{} NEW: also consider compiling\PYZhy{}in the sound\PYZus{}velocity. We will want to see the results of different temperatures and humidities, and sound velocity changes then.}
        \PY{c+c1}{\PYZsh{} also, the code is valid for liquids and solids, so some people will want to use it there.}
        
        \PY{n}{world} \PY{o}{=} \PY{n}{handybeam}\PY{o}{.}\PY{n}{world}\PY{o}{.}\PY{n}{World}\PY{p}{(}\PY{n}{frequency}\PY{o}{=}\PY{l+m+mf}{40e3}\PY{p}{,} \PY{n}{sound\PYZus{}velocity}\PY{o}{=}\PY{l+m+mi}{320}\PY{p}{)}
        
        
        \PY{c+c1}{\PYZsh{} Add a transmitter array to the world}
        \PY{c+c1}{\PYZsh{} NEW: The transmitter comes BEFORE the beamformer. No point in solving the activation if there is no transmitters.}
        \PY{n}{world}\PY{o}{.}\PY{n}{tx\PYZus{}array} \PY{o}{=} \PY{n}{handybeam}\PY{o}{.}\PY{n}{tx\PYZus{}array\PYZus{}library}\PY{o}{.}\PY{n}{rectilinear}\PY{p}{(}\PY{n}{parent} \PY{o}{=} \PY{n}{world}\PY{p}{)}
        
        
        \PY{c+c1}{\PYZsh{}\PYZsh{} NEW: the word \PYZdq{}solver\PYZdq{} is too generic. Try \PYZdq{}beamformer.\PYZdq{}}
        \PY{c+c1}{\PYZsh{} Initialise a beamformer}
        
        \PY{c+c1}{\PYZsh{}!solver = Solver(parent = world)}
        \PY{c+c1}{\PYZsh{} NEW: use full qualification to avoid confusion of where did that came from}
        
        \PY{n}{beamformer} \PY{o}{=} \PY{n}{handybeam}\PY{o}{.}\PY{n}{beamformers}\PY{o}{.}\PY{n}{beamformer}\PY{p}{(}\PY{n}{parent}\PY{o}{=}\PY{n}{world}\PY{p}{)}
        
        
        
        \PY{c+c1}{\PYZsh{} Set the size of the sampling grid (along each axis)}
        \PY{c+c1}{\PYZsh{} New: Do not set the grid size, instead, set the grid spacing (in meters or wavelengths) and extent (in meters). Grid size on it\PYZsq{}}
        \PY{c+c1}{\PYZsh{} justification: the world already knows of the wavelength. the count of points in the grid is not very meaningfull, while grid point spacing\PYZhy{}per\PYZhy{}lambda (default=0.25) is more meaningfull.}
        
        \PY{n}{grid\PYZus{}spacing\PYZus{}per\PYZus{}lambda}\PY{o}{=}\PY{l+m+mf}{0.125} \PY{c+c1}{\PYZsh{} default=0.25}
        \PY{n}{grid\PYZus{}extent\PYZus{}around\PYZus{}the\PYZus{}origin} \PY{o}{=} \PY{l+m+mf}{300e\PYZhy{}3} \PY{c+c1}{\PYZsh{} meaning, 300mm in each direction}
        \PY{c+c1}{\PYZsh{}! N = 100 }
        
        \PY{c+c1}{\PYZsh{} Add a rectilinear sampling grid to the world}
        \PY{n}{basic\PYZus{}sampler} \PY{o}{=} \PY{n}{world}\PY{o}{.}\PY{n}{add\PYZus{}sampler}\PY{p}{(}\PY{n}{handybeam}\PY{o}{.}\PY{n}{samplers}\PY{o}{.}\PY{n}{rectilinear\PYZus{}sampler}\PY{p}{(}\PY{n}{parent}\PY{o}{=}\PY{n}{world}\PY{p}{,}\PY{n}{origin}\PY{o}{=}\PY{n}{something}\PY{p}{,} \PY{n}{rectangular\PYZus{}extent}\PY{o}{=}\PY{l+m+mf}{300e\PYZhy{}3}\PY{p}{,} \PY{n}{density\PYZus{}per\PYZus{}lambda}\PY{o}{=}\PY{l+m+mf}{0.125}\PY{p}{)}\PY{p}{)}
        \PY{c+c1}{\PYZsh{} since the world does know what the frequency and}
        
        \PY{c+c1}{\PYZsh{} Add clist sampler object to the world}
        
        \PY{n}{volume\PYZus{}sampler} \PY{o}{=} \PY{n}{world}\PY{o}{.}\PY{n}{add\PYZus{}sampler}\PY{p}{(}\PY{n}{clist\PYZus{}sampler}\PY{o}{.}\PY{n}{ClistSampler}\PY{p}{(} \PY{n}{parent}\PY{o}{=}\PY{n}{world} \PY{p}{)}\PY{p}{)}
        \PY{n}{sampler2} \PY{o}{=} \PY{n}{world}\PY{o}{.}\PY{n}{add\PYZus{}sampler}\PY{p}{(}\PY{o}{.}\PY{o}{.}\PY{o}{.}\PY{p}{)}
        \PY{n}{sampler3} \PY{o}{=} \PY{n}{world}\PY{o}{.}\PY{n}{add\PYZus{}sampler}\PY{p}{(}\PY{o}{.}\PY{o}{.}\PY{o}{.}\PY{p}{)}
        
        \PY{c+c1}{\PYZsh{} Specify points in the volume to sample the acoustic field on}
        
        \PY{n}{no\PYZus{}points} \PY{o}{=} \PY{l+m+mi}{150}
        
        \PY{n}{x} \PY{o}{=} \PY{n}{np}\PY{o}{.}\PY{n}{linspace}\PY{p}{(}\PY{o}{\PYZhy{}}\PY{l+m+mf}{500e\PYZhy{}3}\PY{p}{,}\PY{l+m+mf}{500e\PYZhy{}3}\PY{p}{,}\PY{n}{no\PYZus{}points}\PY{p}{)}
        \PY{n}{y} \PY{o}{=} \PY{n}{np}\PY{o}{.}\PY{n}{linspace}\PY{p}{(}\PY{o}{\PYZhy{}}\PY{l+m+mf}{500e\PYZhy{}3}\PY{p}{,}\PY{l+m+mf}{500e\PYZhy{}3}\PY{p}{,}\PY{n}{no\PYZus{}points}\PY{p}{)}
        \PY{n}{z} \PY{o}{=} \PY{n}{np}\PY{o}{.}\PY{n}{linspace}\PY{p}{(}\PY{l+m+mf}{10e\PYZhy{}3}\PY{p}{,}\PY{l+m+mf}{500e\PYZhy{}3}\PY{p}{,}\PY{n}{no\PYZus{}points}\PY{p}{)}
        
        \PY{n}{x\PYZus{}mesh}\PY{p}{,}\PY{n}{y\PYZus{}mesh}\PY{p}{,}\PY{n}{z\PYZus{}mesh} \PY{o}{=} \PY{n}{np}\PY{o}{.}\PY{n}{meshgrid}\PY{p}{(}\PY{n}{x}\PY{p}{,}\PY{n}{y}\PY{p}{,}\PY{n}{z}\PY{p}{)}
        
        \PY{n}{x\PYZus{}list} \PY{o}{=} \PY{n}{x\PYZus{}mesh}\PY{o}{.}\PY{n}{ravel}\PY{p}{(}\PY{p}{)}
        \PY{n}{y\PYZus{}list} \PY{o}{=} \PY{n}{y\PYZus{}mesh}\PY{o}{.}\PY{n}{ravel}\PY{p}{(}\PY{p}{)}
        \PY{n}{z\PYZus{}list} \PY{o}{=} \PY{n}{z\PYZus{}mesh}\PY{o}{.}\PY{n}{ravel}\PY{p}{(}\PY{p}{)}
        
        \PY{c+c1}{\PYZsh{} Add these sample points to the sampler}
        
        \PY{n}{volume\PYZus{}sampler}\PY{o}{.}\PY{n}{add\PYZus{}sampling\PYZus{}points}\PY{p}{(}\PY{n}{x\PYZus{}list}\PY{p}{,}\PY{n}{y\PYZus{}list}\PY{p}{,}\PY{n}{z\PYZus{}list}\PY{p}{)}
        
        
        \PY{c+c1}{\PYZsh{} Initialise an evaluator}
        \PY{c+c1}{\PYZsh{} making an object *once* Is OK. }
        \PY{c+c1}{\PYZsh{} the only thing that is not so OK is to make many objects \PYZhy{}\PYZhy{} creating each new object is expensive because there is lots of going on behind the scenes to reserve memory and run lots of conditional code}
        
        \PY{n}{evaluator1} \PY{o}{=} \PY{n}{handybeam}\PY{o}{.}\PY{n}{evaluators}\PY{o}{.}\PY{n}{Evaluator}\PY{p}{(}\PY{n}{rectilinear\PYZus{}sampler}\PY{p}{)}
        \PY{n}{evaluator2} \PY{o}{=} \PY{n}{handybeam}\PY{o}{.}\PY{n}{evaluators}\PY{o}{.}\PY{n}{Evaluator}\PY{p}{(}\PY{n}{rectilinear\PYZus{}sampler}\PY{p}{)}
        \PY{n}{evaluator3} \PY{o}{=} \PY{n}{handybeam}\PY{o}{.}\PY{n}{evaluators}\PY{o}{.}\PY{n}{Evaluator}\PY{p}{(}\PY{n}{rectilinear\PYZus{}sampler}\PY{p}{)}
        \PY{n}{evaluator4} \PY{o}{=} \PY{n}{handybeam}\PY{o}{.}\PY{n}{evaluators}\PY{o}{.}\PY{n}{Evaluator}\PY{p}{(}\PY{n}{rectilinear\PYZus{}sampler}\PY{p}{)}
\end{Verbatim}


    \hypertarget{run-an-example-field}{%
\subsection{\#\# Run an example field}\label{run-an-example-field}}

    \begin{Verbatim}[commandchars=\\\{\}]
{\color{incolor}In [{\color{incolor} }]:} \PY{c+c1}{\PYZsh{} Instruct the beamformer to solve for the activation coefficients}
        
        \PY{n}{beamformer}\PY{o}{.}\PY{n}{single\PYZus{}focus\PYZus{}solver}\PY{p}{(}\PY{n}{x\PYZus{}focus} \PY{o}{=} \PY{l+m+mi}{0}\PY{p}{,} \PY{n}{y\PYZus{}focus} \PY{o}{=} \PY{l+m+mi}{0}\PY{p}{,} \PY{n}{z\PYZus{}focus} \PY{o}{=} \PY{l+m+mf}{200e\PYZhy{}3}\PY{p}{)} \PY{c+c1}{\PYZsh{} Does the beam}
        \PY{c+c1}{\PYZsh{} Propagate the acoustic field}
        
        \PY{n}{world}\PY{o}{.}\PY{n}{propagate}\PY{p}{(}\PY{p}{)}
        
        \PY{n}{handybeam}\PY{o}{.}\PY{n}{evaluators}\PY{o}{.}\PY{n}{present\PYZus{}result}\PY{p}{(}\PY{n}{evaluator1}\PY{p}{)}
        \PY{n}{handybeam}\PY{o}{.}\PY{n}{visualizers}\PY{o}{.}\PY{n}{visualize}\PY{p}{(}\PY{n}{field\PYZus{}1}\PY{p}{)}
        \PY{n}{handybeam}\PY{o}{.}\PY{n}{evaluators}\PY{o}{.}\PY{n}{present\PYZus{}result}\PY{p}{(}\PY{n}{evaluator2}\PY{p}{)}
        \PY{n}{handybeam}\PY{o}{.}\PY{n}{visualizers}\PY{o}{.}\PY{n}{visualize}\PY{p}{(}\PY{n}{field\PYZus{}2}\PY{p}{)}
        \PY{n}{handybeam}\PY{o}{.}\PY{n}{evaluators}\PY{o}{.}\PY{n}{present\PYZus{}result}\PY{p}{(}\PY{n}{evaluator3}\PY{p}{)}
        \PY{n}{handybeam}\PY{o}{.}\PY{n}{visualizers}\PY{o}{.}\PY{n}{visualize}\PY{p}{(}\PY{n}{field\PYZus{}3}\PY{p}{)}
        \PY{n}{handybeam}\PY{o}{.}\PY{n}{evaluators}\PY{o}{.}\PY{n}{present\PYZus{}result}\PY{p}{(}\PY{n}{evaluator4}\PY{p}{)}
        \PY{n}{handybeam}\PY{o}{.}\PY{n}{visualizers}\PY{o}{.}\PY{n}{visualize}\PY{p}{(}\PY{n}{field\PYZus{}4}\PY{p}{)}
\end{Verbatim}


    \hypertarget{run-the-parameter-sweep}{%
\subsection{\#\# Run the parameter
sweep}\label{run-the-parameter-sweep}}

    \begin{Verbatim}[commandchars=\\\{\}]
{\color{incolor}In [{\color{incolor}10}]:} \PY{k+kn}{import} \PY{n+nn}{numpy} \PY{k}{as} \PY{n+nn}{np}
         \PY{n}{parameter\PYZus{}start}\PY{o}{=}\PY{l+m+mf}{100e\PYZhy{}3}
         \PY{n}{parameter\PYZus{}stop}\PY{o}{=}\PY{l+m+mf}{600e\PYZhy{}3}
         \PY{n}{parameter\PYZus{}step}\PY{o}{=}\PY{l+m+mf}{100e\PYZhy{}3}
         
         \PY{n}{parameter\PYZus{}vector}\PY{o}{=}\PY{n}{np}\PY{o}{.}\PY{n}{arange}\PY{p}{(}\PY{n}{parameter\PYZus{}start}\PY{p}{,}\PY{n}{parameter\PYZus{}stop}\PY{p}{,}\PY{n}{parameter\PYZus{}step}\PY{p}{,}\PY{n}{dtype}\PY{o}{=}\PY{n}{np}\PY{o}{.}\PY{n}{float32}\PY{p}{)}
         \PY{n}{step\PYZus{}count}\PY{o}{=}\PY{n}{parameter\PYZus{}vector}\PY{o}{.}\PY{n}{size}
         \PY{n+nb}{print}\PY{p}{(}\PY{l+s+s1}{\PYZsq{}}\PY{l+s+s1}{got }\PY{l+s+si}{\PYZob{}\PYZcb{}}\PY{l+s+s1}{ steps to do}\PY{l+s+s1}{\PYZsq{}}\PY{o}{.}\PY{n}{format}\PY{p}{(}\PY{n}{step\PYZus{}count}\PY{p}{)}\PY{p}{)}
\end{Verbatim}


    \begin{Verbatim}[commandchars=\\\{\}]
got 5 steps to do

    \end{Verbatim}

    \begin{Verbatim}[commandchars=\\\{\}]
{\color{incolor}In [{\color{incolor} }]:} \PY{c+c1}{\PYZsh{} initialize the data storage}
        \PY{n}{metrics\PYZus{}storage1}\PY{o}{=}\PY{n}{np}\PY{o}{.}\PY{n}{NaN}\PY{p}{(}\PY{n}{step\PYZus{}count}\PY{p}{,}\PY{n}{evaluator1}\PY{o}{.}\PY{n}{result\PYZus{}size}\PY{p}{)}
        \PY{n}{metrics\PYZus{}storage2}\PY{o}{=}\PY{n}{np}\PY{o}{.}\PY{n}{NaN}\PY{p}{(}\PY{n}{step\PYZus{}count}\PY{p}{,}\PY{n}{evaluator2}\PY{o}{.}\PY{n}{result\PYZus{}size}\PY{p}{)}
        \PY{n}{metrics\PYZus{}storage3}\PY{o}{=}\PY{n}{np}\PY{o}{.}\PY{n}{NaN}\PY{p}{(}\PY{n}{step\PYZus{}count}\PY{p}{,}\PY{n}{evaluator3}\PY{o}{.}\PY{n}{result\PYZus{}size}\PY{p}{)}
        \PY{n}{metrics\PYZus{}storage4}\PY{o}{=}\PY{n}{np}\PY{o}{.}\PY{n}{NaN}\PY{p}{(}\PY{n}{step\PYZus{}count}\PY{p}{,}\PY{n}{evaluator4}\PY{o}{.}\PY{n}{result\PYZus{}size}\PY{p}{)}
\end{Verbatim}


    \begin{Verbatim}[commandchars=\\\{\}]
{\color{incolor}In [{\color{incolor}12}]:} \PY{k}{for} \PY{p}{(}\PY{n}{step\PYZus{}idx}\PY{p}{,}\PY{n}{parameter}\PY{p}{)} \PY{o+ow}{in} \PY{n+nb}{zip}\PY{p}{(}\PY{n+nb}{range}\PY{p}{(}\PY{n}{parameter\PYZus{}vector}\PY{o}{.}\PY{n}{size}\PY{p}{)}\PY{p}{,} \PY{n}{parameter\PYZus{}vector}\PY{p}{)} \PY{p}{:}
             \PY{n+nb}{print}\PY{p}{(}\PY{l+s+s1}{\PYZsq{}}\PY{l+s+s1}{step }\PY{l+s+si}{\PYZob{}:03d\PYZcb{}}\PY{l+s+s1}{: parameter=}\PY{l+s+si}{\PYZob{}:0.2f\PYZcb{}}\PY{l+s+s1}{mm}\PY{l+s+s1}{\PYZsq{}}\PY{o}{.}\PY{n}{format}\PY{p}{(}\PY{n}{step\PYZus{}idx}\PY{p}{,}\PY{n}{parameter}\PY{o}{*}\PY{l+m+mf}{1e3}\PY{p}{)}\PY{p}{)}
\end{Verbatim}


    \begin{Verbatim}[commandchars=\\\{\}]
step 000: parameter=100.00mm
step 001: parameter=200.00mm
step 002: parameter=300.00mm
step 003: parameter=400.00mm
step 004: parameter=500.00mm

    \end{Verbatim}

    \begin{Verbatim}[commandchars=\\\{\}]
{\color{incolor}In [{\color{incolor}11}]:} \PY{k}{for} \PY{p}{(}\PY{n}{step\PYZus{}idx}\PY{p}{,}\PY{n}{parameter}\PY{p}{)} \PY{o+ow}{in} \PY{n+nb}{zip}\PY{p}{(}\PY{n+nb}{range}\PY{p}{(}\PY{n}{parameter\PYZus{}vector}\PY{o}{.}\PY{n}{size}\PY{p}{)}\PY{p}{,} \PY{n}{parameter\PYZus{}vector}\PY{p}{)} \PY{p}{:}
             \PY{n+nb}{print}\PY{p}{(}\PY{l+s+s1}{\PYZsq{}}\PY{l+s+s1}{step }\PY{l+s+si}{\PYZob{}:03d\PYZcb{}}\PY{l+s+s1}{: parameter=}\PY{l+s+si}{\PYZob{}:0.2f\PYZcb{}}\PY{l+s+s1}{mm}\PY{l+s+s1}{\PYZsq{}}\PY{o}{.}\PY{n}{format}\PY{p}{(}\PY{n}{step\PYZus{}idx}\PY{p}{,}\PY{n}{parameter}\PY{o}{*}\PY{l+m+mf}{1e3}\PY{p}{)}\PY{p}{)}
             
             \PY{c+c1}{\PYZsh{} Instruct the beamformer to solve for the activation coefficients}
             \PY{n}{beamformer}\PY{o}{.}\PY{n}{single\PYZus{}focus\PYZus{}solver}\PY{p}{(}\PY{n}{x\PYZus{}focus} \PY{o}{=} \PY{l+m+mi}{0}\PY{p}{,} \PY{n}{y\PYZus{}focus} \PY{o}{=} \PY{l+m+mi}{0}\PY{p}{,} \PY{n}{z\PYZus{}focus} \PY{o}{=} \PY{n}{parameter}\PY{p}{)} \PY{c+c1}{\PYZsh{} Does the beam}
             
             \PY{c+c1}{\PYZsh{} Propagate the acoustic field}
             \PY{n}{world}\PY{o}{.}\PY{n}{propagate}\PY{p}{(}\PY{p}{)}
             
             \PY{c+c1}{\PYZsh{} get the results}
             \PY{n}{metrics\PYZus{}storage1}\PY{p}{(}\PY{n}{step\PYZus{}idx}\PY{p}{,}\PY{p}{:}\PY{p}{)}\PY{o}{=}\PY{n}{evaluator1}\PY{o}{.}\PY{n}{results}
             \PY{n}{metrics\PYZus{}storage2}\PY{p}{(}\PY{n}{step\PYZus{}idx}\PY{p}{,}\PY{p}{:}\PY{p}{)}\PY{o}{=}\PY{n}{evaluator2}\PY{o}{.}\PY{n}{results}
             \PY{n}{metrics\PYZus{}storage3}\PY{p}{(}\PY{n}{step\PYZus{}idx}\PY{p}{,}\PY{p}{:}\PY{p}{)}\PY{o}{=}\PY{n}{evaluator3}\PY{o}{.}\PY{n}{results}
             \PY{n}{metrics\PYZus{}storage4}\PY{p}{(}\PY{n}{step\PYZus{}idx}\PY{p}{,}\PY{p}{:}\PY{p}{)}\PY{o}{=}\PY{n}{evaluator1}\PY{o}{.}\PY{n}{results}
             
             \PY{c+c1}{\PYZsh{} save the visualisation result}
             \PY{n}{filename\PYZus{}1}\PY{o}{=}\PY{l+s+s1}{\PYZsq{}}\PY{l+s+se}{\PYZbs{}r}\PY{l+s+s1}{esult\PYZus{}vis}\PY{l+s+se}{\PYZbs{}f}\PY{l+s+s1}{ield1}\PY{l+s+se}{\PYZbs{}f}\PY{l+s+s1}{1}\PY{l+s+si}{\PYZob{}:03d\PYZcb{}}\PY{l+s+s1}{.png}\PY{l+s+s1}{\PYZsq{}}\PY{o}{.}\PY{n}{format}\PY{p}{(}\PY{n}{step\PYZus{}idx}\PY{p}{)}
             \PY{n}{handybeam}\PY{o}{.}\PY{n}{visualizers}\PY{o}{.}\PY{n}{visualize\PYZus{}save}\PY{p}{(}\PY{n}{field\PYZus{}1}\PY{p}{,}\PY{n}{filename\PYZus{}1}\PY{p}{)}
             
             \PY{n}{filename\PYZus{}2}\PY{o}{=}\PY{l+s+s1}{\PYZsq{}}\PY{l+s+se}{\PYZbs{}r}\PY{l+s+s1}{esult\PYZus{}vis}\PY{l+s+se}{\PYZbs{}f}\PY{l+s+s1}{ield2}\PY{l+s+se}{\PYZbs{}f}\PY{l+s+s1}{2}\PY{l+s+si}{\PYZob{}:03d\PYZcb{}}\PY{l+s+s1}{.png}\PY{l+s+s1}{\PYZsq{}}\PY{o}{.}\PY{n}{format}\PY{p}{(}\PY{n}{step\PYZus{}idx}\PY{p}{)}
             \PY{n}{handybeam}\PY{o}{.}\PY{n}{visualizers}\PY{o}{.}\PY{n}{visualize\PYZus{}save}\PY{p}{(}\PY{n}{field\PYZus{}2}\PY{p}{,}\PY{n}{filename\PYZus{}2}\PY{p}{)}
             
             \PY{c+c1}{\PYZsh{} save field for further analysis later on}
             \PY{n}{filename\PYZus{}fieldsaver}\PY{o}{=}\PY{l+s+s1}{\PYZsq{}}\PY{l+s+s1}{result\PYZus{}fields}\PY{l+s+se}{\PYZbs{}f}\PY{l+s+s1}{1}\PY{l+s+si}{\PYZob{}:03d\PYZcb{}}\PY{l+s+s1}{.h5}\PY{l+s+s1}{\PYZsq{}}\PY{o}{.}\PY{n}{format}\PY{p}{(}\PY{n}{step\PYZus{}idx}\PY{p}{)}
             \PY{n}{hf\PYZus{}field} \PY{o}{=} \PY{n}{h5py}\PY{o}{.}\PY{n}{File}\PY{p}{(}\PY{n}{filename\PYZus{}fieldsaver}\PY{p}{,} \PY{l+s+s1}{\PYZsq{}}\PY{l+s+s1}{w}\PY{l+s+s1}{\PYZsq{}}\PY{p}{)}
             \PY{n}{hf\PYZus{}field}\PY{o}{.}\PY{n}{create\PYZus{}dataset}\PY{p}{(}\PY{l+s+s1}{\PYZsq{}}\PY{l+s+s1}{field\PYZus{}raw}\PY{l+s+s1}{\PYZsq{}}\PY{p}{,}\PY{n}{data}\PY{o}{=}\PY{n}{sampler1}\PY{o}{.}\PY{n}{p}\PY{p}{)}
             \PY{n}{hf\PYZus{}field}\PY{o}{.}\PY{n}{close}\PY{p}{(}\PY{p}{)}
             
             
             
         \PY{n}{hf\PYZus{}metrics} \PY{o}{=} \PY{n}{h5py}\PY{o}{.}\PY{n}{File}\PY{p}{(}\PY{l+s+s1}{\PYZsq{}}\PY{l+s+s1}{data.h5}\PY{l+s+s1}{\PYZsq{}}\PY{p}{,} \PY{l+s+s1}{\PYZsq{}}\PY{l+s+s1}{w}\PY{l+s+s1}{\PYZsq{}}\PY{p}{)}
         \PY{n}{hgroup}\PY{o}{=}\PY{n}{hf\PYZus{}metrics}\PY{o}{.}\PY{n}{create\PYZus{}group}\PY{p}{(}\PY{l+s+s1}{\PYZsq{}}\PY{l+s+s1}{metrics}\PY{l+s+s1}{\PYZsq{}}\PY{p}{)}
         \PY{n}{hgroup}\PY{o}{.}\PY{n}{create\PYZus{}dataset}\PY{p}{(}\PY{l+s+s1}{\PYZsq{}}\PY{l+s+s1}{field\PYZus{}1\PYZus{}sweep}\PY{l+s+s1}{\PYZsq{}}\PY{p}{,}\PY{n}{data}\PY{o}{=}\PY{n}{metrics\PYZus{}storage1}\PY{p}{)}
         \PY{n}{hgroup}\PY{o}{.}\PY{n}{create\PYZus{}dataset}\PY{p}{(}\PY{l+s+s1}{\PYZsq{}}\PY{l+s+s1}{field\PYZus{}2\PYZus{}sweep}\PY{l+s+s1}{\PYZsq{}}\PY{p}{,}\PY{n}{data}\PY{o}{=}\PY{n}{metrics\PYZus{}storage2}\PY{p}{)}
         \PY{n}{hgroup}\PY{o}{.}\PY{n}{create\PYZus{}dataset}\PY{p}{(}\PY{l+s+s1}{\PYZsq{}}\PY{l+s+s1}{field\PYZus{}3\PYZus{}sweep}\PY{l+s+s1}{\PYZsq{}}\PY{p}{,}\PY{n}{data}\PY{o}{=}\PY{n}{metrics\PYZus{}storage3}\PY{p}{)}
         \PY{n}{hgroup}\PY{o}{.}\PY{n}{create\PYZus{}dataset}\PY{p}{(}\PY{l+s+s1}{\PYZsq{}}\PY{l+s+s1}{field\PYZus{}4\PYZus{}sweep}\PY{l+s+s1}{\PYZsq{}}\PY{p}{,}\PY{n}{data}\PY{o}{=}\PY{n}{metrics\PYZus{}storage4}\PY{p}{)}
         \PY{n}{hf}\PY{o}{.}\PY{n}{close}\PY{p}{(}\PY{p}{)}
\end{Verbatim}


    \begin{Verbatim}[commandchars=\\\{\}]
step 0: parameter=100.00mm
step 1: parameter=200.00mm
step 2: parameter=300.00mm
step 3: parameter=400.00mm
step 4: parameter=500.00mm

    \end{Verbatim}

    \hypertarget{preview-the-result}{%
\subsection{\#\# Preview the result}\label{preview-the-result}}

    \begin{Verbatim}[commandchars=\\\{\}]
{\color{incolor}In [{\color{incolor} }]:} \PY{n}{hf}\PY{o}{=}\PY{n}{plt}\PY{o}{.}\PY{n}{figure}\PY{p}{(}\PY{n}{figsize}\PY{o}{=}\PY{p}{(}\PY{n}{something}\PY{p}{)}\PY{p}{)}
        \PY{n}{hf}\PY{o}{.}\PY{n}{plot}\PY{p}{(}\PY{n}{parameter\PYZus{}vector}\PY{p}{,}\PY{n}{storage1}\PY{p}{(}\PY{p}{:}\PY{p}{,}\PY{l+m+mi}{3}\PY{p}{)}\PY{p}{)}
        \PY{n}{hf}\PY{o}{.}\PY{n}{xlabel}\PY{p}{(}\PY{l+s+s1}{\PYZsq{}}\PY{l+s+s1}{parameter}\PY{l+s+s1}{\PYZsq{}}\PY{p}{)}
        \PY{n}{hf}\PY{o}{.}\PY{n}{ylabel}\PY{p}{(}\PY{l+s+s1}{\PYZsq{}}\PY{l+s+s1}{metric 3}\PY{l+s+s1}{\PYZsq{}}\PY{p}{)}
        \PY{n}{hf}\PY{o}{.}\PY{n}{title}\PY{p}{(}\PY{l+s+s1}{\PYZsq{}}\PY{l+s+s1}{metric 3 vs parameter on field 1}\PY{l+s+s1}{\PYZsq{}}\PY{p}{)}
\end{Verbatim}


    \begin{Verbatim}[commandchars=\\\{\}]
{\color{incolor}In [{\color{incolor} }]:} \PY{c+c1}{\PYZsh{}\PYZsh{} Analise the result . . . }
        \PY{c+c1}{\PYZsh{} load the data from file to avoid the need to re\PYZhy{}run . . . }
        \PY{c+c1}{\PYZsh{} see http://docs.h5py.org/en/stable/high/dataset.html\PYZsh{}reading\PYZhy{}writing\PYZhy{}data}
        
        \PY{n}{hf\PYZus{}field} \PY{o}{=} \PY{n}{h5py}\PY{o}{.}\PY{n}{File}\PY{p}{(}\PY{n}{filename\PYZus{}fieldsaver}\PY{p}{,} \PY{l+s+s1}{\PYZsq{}}\PY{l+s+s1}{r}\PY{l+s+s1}{\PYZsq{}}\PY{p}{)}
        \PY{n}{hfield\PYZus{}data} \PY{o}{=} \PY{n}{hf\PYZus{}field}\PY{o}{.}\PY{n}{create\PYZus{}dataset}\PY{p}{(}\PY{l+s+s1}{\PYZsq{}}\PY{l+s+s1}{field\PYZus{}raw}\PY{l+s+s1}{\PYZsq{}}\PY{p}{)}
        \PY{c+c1}{\PYZsh{} access data}
        \PY{n}{hfield\PYZus{}data}\PY{p}{[}\PY{p}{:}\PY{p}{,}\PY{p}{:}\PY{p}{]}
\end{Verbatim}



    % Add a bibliography block to the postdoc
    
    
    
    \end{document}
